\documentclass[a4J, 12pt]{jsarticle}
\usepackage{additional}

\title{プログラミング言語実験 javascript編\\課題5レポート}
\author{s1710730\\須藤敬仁}
\date{}

\begin{document}
\maketitle
%=====================================%
\section{生命体の消滅}
5*5の正方形の対角線となるように生命体が存在している場合、3世代目で生命体が消滅する.
例えば図\ref{fig:life}の様な初期状態は3世代目ですべての生命体が消滅する.
\begin{figure}[tb] 
    \centering
    \begin{tabular}{|p{1em}|p{1em}|p{1em}|p{1em}|p{1em}|} \hline    
        ● & & & & \\ \hline
        & ● & & & \\ \hline
        & & ● & & \\ \hline
        & & & ● & \\ \hline
        & & & & ● \\ \hline
    \end{tabular}
    \caption{3世代目で生命体が消滅する初期状態の例}
    \label{fig:life}
\end{figure}
%=====================================%
\section{繰り返し}
%-------------------------------------%
\subsection{横3つもしくは縦3つの生命体がある場合}
横3つと縦3つの中央の生命体を軸として90度回転を繰り返すような状態が存在する. ただし、端
に存在していて回転の余地が無い場合は消滅する. 例を図\ref{fig:repeat1_1}, \ref{fig:repeat1_2}に示す.
\begin{figure}[tbp]
    \centering
    \begin{tabular}{cc}
        \begin{minipage}[c]{0.5\hsize}
            \centering
            \begin{tabular}{|p{1em}|p{1em}|p{1em}|p{1em}|p{1em}|} \hline
                & & & & \\ \hline
                & ● & ● & ● & \\ \hline
                & & & & \\ \hline
                & & & & \\ \hline
                & & & & \\ \hline
            \end{tabular}
            \caption{繰り返しの例1:奇数状態}
            \label{fig:repeat1_1}
        \end{minipage}
        %
        \begin{minipage}[c]{0.5\hsize}
            \centering
            \begin{tabular}{|p{1em}|p{1em}|p{1em}|p{1em}|p{1em}|} \hline
                & & ● & & \\ \hline
                & & ● & & \\ \hline
                & & ● & & \\ \hline
                & & & & \\ \hline
                & & & & \\ \hline
            \end{tabular}
            \caption{繰り返しの例1:偶数状態}
            \label{fig:repeat1_2}
        \end{minipage}
    \end{tabular}
\end{figure}

%-------------------------------------%
\subsection{正方形の4辺の中央に3つ生命体が並んでいる場合}
図\ref{fig:repeat2_1}と図\ref{fig:repeat2_2}は繰り返す.
\begin{figure}[tbp]
    \centering
    \begin{tabular}{cc}
        \begin{minipage}[c]{0.5\hsize}
            \centering
            \begin{tabular}{|p{1em}|p{1em}|p{1em}|p{1em}|p{1em}|} \hline
                  & ● & ● & ● & \\ \hline
                ● &   &   &   & ● \\ \hline
                ● &   &   &   & ● \\ \hline
                ● &   &   &   & ● \\ \hline
                  & ● & ● & ● & \\ \hline
            \end{tabular}
            \caption{繰り返しの例2:奇数状態}
            \label{fig:repeat2_1}
        \end{minipage}
        %
        \begin{minipage}[c]{0.5\hsize}
            \centering
            \begin{tabular}{|p{1em}|p{1em}|p{1em}|p{1em}|p{1em}|} \hline
                  & ● & ● & ● & \\ \hline
                ● &   & ● &   & ● \\ \hline
                ● & ● &   & ● & ● \\ \hline
                ● &   & ● &   & ● \\ \hline
                  & ● & ● & ● & \\ \hline
            \end{tabular}
            \caption{繰り返しの例2:偶数状態}
            \label{fig:repeat2_2}
        \end{minipage}
    \end{tabular}
\end{figure}
\end{document}